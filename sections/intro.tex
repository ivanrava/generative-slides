\begin{frame}
    \frametitle{Goal}
    The following slides aim to provide a compact overview of the most popular \textbf{Deep Learning approaches and techniques for image data generation} (often also simply called \emph{generative approaches}), both from a \emph{theoretical} and \emph{practical} standpoint.

    \begin{block}{On the task of image generation}
        While most of the examples here cover ``convolutional-flavors'' of the proposed architectures, which are mainly geared towards image generation, these approaches can still be used for other generative tasks, often with big success (like text generation, speech synthesis, etc.)
    \end{block}

\end{frame}

\begin{frame}
    \frametitle{Structure of the document}
    For every architecture the slides present first a quick theoretical introduction and next a heavily commented Python notebook with a demonstration of the previously covered DL model using common datasets (like MNIST digits, MNIST fashion and CelebA).

    In the end, I propose a final series of Python notebooks which compare the generative performances of the aforementioned architectures using a dataset found ``in the wild'', while showcasing all the needed preprocessing, fixes, analyses and tuning conducted during a typical ML/DL pipeline.
\end{frame}

\begin{frame}
    \frametitle{Topics / covered architectures}
    \tableofcontents
\end{frame}